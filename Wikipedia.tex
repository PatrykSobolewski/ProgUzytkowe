\documentclass[a4paper,12pt]{article}
\usepackage[MeX]{polski}
\usepackage[utf8]{inputenc}
\usepackage{graphicx}
%opening
\title{World Monuments Fund}
\author{Patryk Sobolewski}

\begin{document}

\maketitle

\begin{abstract}



\end{abstract}


World Monuments Fund (WMF) – międzynarodowa organizacja non-profit, której działalność skierowana jest na ochronę światowej architektury historycznej i dziedzictwa kulturowego poprzez pracę w terenie, adwokaturę, przekazywanie dotacji, edukację i szkolenie.

Organizacja powstała w 1965 roku. Siedziba znajduje się w Nowym Jorku, a biura i oddziały zlokalizowane są na całym świecie, m.in. we Włoszech, Francji, Peru, Portugalii, Hiszpanii i Wielkiej Brytanii. Oddziały identyfikują, rozwijają i kierują projektami, negocjują z lokalnymi partnerami oraz szukają miejscowego wsparcia w celu uzupełnienia wcześniej otrzymanych darowizn.
\begin{figure}
\includegraphics[width=0.25\hsize] {teatr.jpg}
\caption{Teatr Stary w Lublinie przed renowacją.}\label{fig:teatr}
\end{figure}

\section{World Monuments Watch}

Od 1996 roku – co dwa lata – World Monuments Watch działa na rzecz światowego dziedzictwa kulturowego, gdzie istnieje ryzyko spowodowane czynnikami naturalnymi lub z tytułu zmian socjalnych, politycznych oraz ekonomicznych.

\begin{table}
\begin{tabular}{lc}
\hline
\textbf{Polskie zabytki na liście}&\textbf{Rok budowy}\\
\hline
drewniany kościół św.Michała Archanioła w Dębnie Podhalańskim&1996{,}1998\\
kościół Wniebowzięcia NMP w Hebdowie&1996\\
ul. Próżna w Warszawie&1996\\
Twierdza Wisłoujście w Gdańsku&1998{,}2000\\
stanowisko archeologiczne w Wiślicy&2002\\
Teatr Stary w Lublinie&2004\\
Szpital Jerozolimski w Malborku&2006\\
cerkiew św. Paraskewy w Radrużu&2012\\

\hline
\end{tabular}
\end{table}

\section{Przypisy}
About Us {/} World Monuments Fund\\
Affiliates {/} World Monuments Fund\\
The Watch {/} World Monuments Fund\\

\section{Linki zewnętrzne}
Strona WMF




\end{document}