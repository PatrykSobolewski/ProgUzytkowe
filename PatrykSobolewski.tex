\documentclass[a4paper,12pt]{article}
\usepackage[MeX]{polski}
\usepackage[utf8]{inputenc}

%opening
\title{Wydział Matematyki i Inforamtyki Uniwersytetu Warmińsko-Mazurskiego}
\author{Patryk Sobolewski}

\begin{document}

\maketitle
\begin{abstract}
Wydział Matematyki i Inforamtyki Uniwersystetu Warmińsko-Mazurskiego

(WMiI)-wydział Uniwersytetu Warmińsko-Mazurskiego w Olsztynie oferujący studia na dwóch kierunkach:

Informatyka
Matematyka

w trybie studiów stacjonarnych i niestacjonarnych. Ponadto oferuje studia podyplomowe.

Wydział zatrudnia 8 profesorów, 14 doktorów habilitowanych, 53doktorów i 28 magistrów.

\end{abstract}
%pierwsza sekcja
\section{Misja}
Misja Wydziału jest:

Kształcenie matematyków zdolnych do udziału w rozwijaniu matematyki i jej stosowania w innych działach wiedzy i w praktyce;

Kształcenie nauczycieli matematyki, nauczycieli matematyki z fizyką a także nauczycieli informatyki;

Kształcenie profesjonalnych informatyków dla potrzeb gospodarki, administracji, szkolnictwa oraz życia
społecznego;

Nauczanie matematyki i jej działów specjalnych jak statystyka matematyczna, ekonometria,
biomatematyka, ekologia matematyczna, metody numeryczne; fizyki a w razie potrzeby i podstaw
informatyki na wszystkich wydziałach UWM.

%druga sekcja
\section{Opis kierunków}
Na kierunku Informatyka prowadzone są studia stacjonarne i niestacjonarne:

studia pierwszego stopnia – inżynierskie (7 sem.), sp. inżynieria systemów informatycznych, informatyka
ogólna


studia drugiego stopnia – magisterskie (4 sem.), sp. techniki multimedialne, projektowanie systemów
informatycznych i sieci komputerowych



Na kierunku Matematyka prowadzone są studia stacjonarne:

studia pierwszego stopnia – licencjackie (6 sem.), sp. nauczanie matematyki, matematyka stosowana

studia drugiego stopnia – magisterskie (4 sem.), sp. nauczanie matematyki, matematyka stosowana



\end{document}