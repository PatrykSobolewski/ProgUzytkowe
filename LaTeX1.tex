\documentclass[a4paper,12pt]{article}
\usepackage[MeX]{polski}
\usepackage[utf8]{inputenc}

%opening
\title{Formuły Matematyczne TEX}
\author{Patryk Sobolewski}

\begin{document}

\maketitle

Ułamek w tekście $\frac{1}{x} $ 
Oto równanie $c^{2}=a^{2}+b^{2}$

Ułamek $$\frac{1}{x} $$ 
Oto równanie $$c^{2}=a^{2}+b^{2}$$



Ułamek

\begin{equation}
\frac{1}{x}
\label{eq:równanie1}
\end{equation}

Oto równanie

\begin{equation}
$$c^{2}=a^{2}+b^{2}$$
\label{eq:równanie2}
\end{equation}

Indeks górny $$x^{y}  \  e^{x}  \  2^{e}  \  A^{2 \times 2}$$
Indeks dolny $$ x_y  \  a_{ij} $$
$$ \frac{2^{k}}{2^{k + 2}} $$
$$  2^{\frac{x^{2}}{(x+2)(x-3)^{3}}}$$
$$  \vec{x}

$$\sum \ \sum_{i=1}^{10}x_{i} \ \prod \ \coprod \ \int \ \oint \ \bigcap \ \bigcup \ \bigsqcup \ \bigvee \ \bigwedge \ \bigodot \ \bigotimes \ \bigoplus \ \biguplus$$

$$\hat{a} \ \check{b} \ \breve{c} \ \acute{d} \ \grave{e} \ \tilde{f} \ \bar{g} \ \vec{h} \ \dot{m} \ \ddot{n}$$

$$\widetilde{aaa} \ \widehat{bbb} \ \overleftarrow{ccc} \ \overrightarrow{ddd} \ \overline{eee} \ \overbrace{fff} \ \underbrace{ggg} \ \underline{hhh} \ \sqrt{iii} \ \sqrt[n]{jjj} \ \frac{kkkk}{}$$

$$\Gamma \ \Delta \ \Theta \ \Xi \ \Pi \ \Sigma \ \Upsilon \ \Phi \ \Psi \ \Omega$$

\section{}

\end{document}